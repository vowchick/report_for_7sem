\documentclass[12pt]{article}
\usepackage{lipsum}

\makeatletter
\def\CTEX@section@format{\Large\bfseries}
\makeatother

\usepackage[T2A]{fontenc}
\usepackage[utf8]{inputenc}
\usepackage{diagbox}
\usepackage{amsmath}
\usepackage[english, russian]{babel}
\usepackage{listings}


\begin{document}
\begin{center} \large \bf
Мехмат. Сошников Владимир, группа 410.\\
Численное моделирование нестационарного одномерного течения газа с использованием неявной последовательной разностной схемы А.Г. Соколова   ($u$, $\rho$). 
\end{center}

\section{Постановка задачи}

В данной работе будет рассматриваться разностная схема А.Г. Соколова для решения начально - краевых задач для системы уравнений, описывающей нестационарное одномерное движение вязкого баротропного газа:

\begin{equation*}
 \begin{cases}
   \text
     {
       $
         \frac {\partial \rho} {\partial t} + 
         \frac {\partial \rho u} {\partial x} = 
         0
       $
     }
   \\
   \text 
     {
       $
          \rho \frac {\partial u} {\partial t} +
          \rho u \frac {\partial u} {\partial x} + 
          \frac {\partial p} {\partial x} =
          \mu \frac {\partial^{2} u} {\partial x ^{2}} +
          \rho f
       $
     }
   \\\tag{*}
   \text 
     {
       $
         p = p (\rho)
       $
     } 
   \\
   \text 
     {
       $
        (\rho, u)|_{t = 0} = (\rho_{0}, \,u_{0}), \,  x \in [0, X]
       $
     } 
   \\
   \text 
     {
       $
        u (t, 0) = u (t, X) = 0, \,t \in [0, T]
       $
     } 
 \end{cases}
\end{equation*}
Неизвестные функции: плотность $\rho$ и скорость $u$ являются функциями переменных Эйлера
$$
(t,x) \in Q = [0,T] \times [0,X].
$$
В уравнения входят еще две неизвестные функции - давление$p$, зависящее от плотности и вектор внешних сил $f$, являющийся функцией переменных Эйлера.
В данной работе рассматривается зависимость $p = \rho ^{\gamma}$, где $\gamma = 1.4$
\section{Описание алгоритма}
\subsection{Используемые обозначения}
Рассмотрим временной интервал $[0,T]$, введем на нем равномерную сетку:
$ \bar w_{\tau} = \{n \tau | n = 0,...,N\}$, $N \tau = T$
Рассмотрим пространственную область в виде отрезка $\bar \Omega = [0, X]$. Обозначим  через $\bar w_{h} = \{mh|m = 0,...,M\}$ (где $Mh = X$), равномерную сетку с шагом $h$, а
через $w_{h}^{1/2} = \{mh + \frac{h}{2} | m = 0,..., M - 1\}$ сдвинутую равномерную сетку с полуцелыми узлами.
Введем также обозначения для сеток:
$
\bar Q_{\tau h} := \bar \omega_{\tau} \times \bar \omega_{h}
$
и 
$
Q_{\tau h}^{1/2} = \bar \omega_{\tau} \times \omega_{h}^{1/2}
$

Значение функции $g$ в узле (m, n) будем обозначать через $g_{n}^{m}$
Для сокращения записей будем также использовать следующие обозначения:
\\
$
g_{m}^{n+1} = \hat g \
$,
$
g_{m \pm 1} ^{n} = g^{\pm 1}
$

Обозначения для среднего значения величин сеточной функции в двух соседних узлах: \\
$
g_{s} = \frac{g_{m+1}^{n} + g_{m}^{n}}{2}
$,
$
g_{\bar s} = \frac {g_{m}^{n} + g_{m-1}^{n}}{2}
$
\\
Для разностных операторов применяются следующие обозначения:
\\
$$
g_{t}=\frac{g_{m}^{n+1}-g_{m}^{n}}{\tau}, \quad g_{x}=\frac{g_{m+1}^{n}-g_{m}^{n}}{h},  \quad g_{\bar{x}}=\frac{g_{m}^{n}-g_{m-1}^{n}}{h}, \quad g_{x \bar{x}}=\frac{g_{m-1}^{n}-2 g_{m}^{n}+g_{m+1}^{n}}{h^{2}}
$$
Для приближения конвективных слагаемых в дифференциальных операторах будут использоваться разностные аппроксимации 
против потока. Для этих выражений введем обозначение:
$$
\delta\{W, V\}=\frac{V+|V|}{2} W_{\bar{x}}+\frac{V-|V|}{2} W_{x}=\left\{\begin{array}{l}
V W_{\bar{x}}, \text { если } V \geq 0 \\
V W_{x}, \text { если } V<0
\end{array}\right.
$$
В данной схеме в конвективных слагаемых узел шаблона, в котором нужно брать значение сеточной функции $F$,
 зависит от знака компоненты вектора скорости $V$. Для этих выражений используется обозначение:
$$
\sigma\{F, V\}=F \frac{|V|-V}{2|V|}+F^{(-1)} \frac{|V|+V}{2|V|}=\left\{\begin{array}{l}
F, \text { ecли } V<0 \\
F^{(-1)}, \text {если } V \geq 0
\end{array}\right.
$$

\subsection{Схема в веденных обозначениях}

    $ \left\{ \begin{array}{l}
H_{ \bar s} V_{t}+H_{\bar s} \delta\{\hat{V}, V\}+\frac{\gamma}{\gamma-1} H_{\bar{s}}\left((H)^{\gamma-1}\right)_{\bar x}=\mu \hat{V}_{x \bar{x}}+H_{\bar{s}} f, \quad \text { при } H_{\bar{s}} \neq 0 \quad (1)\\ \\
\hat{V}=0, \quad \text { при } H_{\bar{s}}=0 \quad (2) \\ \\
0<m<M, n \geq 0 \\ \\
\hat{V}_{0}=\hat{V}_{M}=0, \quad (3)\\ \\
H_{t}+(\sigma\{\hat{H}, \hat{V}\} \hat{V})_{x}=0,  \quad 0 \leq m<M, n \geq 0 \quad (4)
\end{array}
\right.
$

\subsection{Поточечная запись схемы}
$
  \left \{
    \begin{array}{l}
      V_{m-1}^{n+1} (-\frac{\tau}{4h} (V_{m}^{n} + |V_{m}^{n}|) (H_{m}^{n} + H_{m - 1} ^{n}) - \frac {\tau \mu} {h^{2}}) \, + \\ \\ 
      + \,V_{m}^{n + 1} (\frac{H_{m}^{n} + H_{m - 1} ^ {n}}{2} + \frac{2 \tau \mu} {h^{2}} + \frac{\tau}{2h} |V_{m}^{n}|(H_{m}^{n} + H_{m -  1} ^ {n})) +\\ \\
      + \,V_{m+1}^{n+1} (\frac {\tau}{4h} (V_{m}^{n} - |V_{m}^{n}|) (H_{m}^{n} + H_{m-1}^{n}) - \frac{\tau \mu} {h^{2}})  = \\   \\
      =\, V_{m}^{n} \frac{H_{m}^{n} + H_{m - 1}^{n}} {2} - \frac {\tau \gamma}{2h (\gamma - 1)} (H_{m}^{n} + H_{m - 1}^{n}) ((H_{m}^{n})^{\gamma - 1} - 
      (H_{m - 1}^{n})^{\gamma - 1}) +  \\ \\ 
      + \,\tau f_{m}^{n} \frac{H_{m}^{n} + H_{m-1}^{n}}{2}, \quad \text{при} \, \frac{H_{m}^{n} + H_{m - 1}^{n}}{2} \neq 0 \quad (1) \\ \\
      0<m<M, n \geq 0 \\ \\
      V_{m}^{n+1} = 0, \quad \text{при} \, \frac{H_{m}^{n} + H_{m - 1}^{n}}{2} = 0 \quad (2)\\ \\
      V_{0}^{n+1} = V_{M}^{n+1} = 0 \quad(3) \\ \\ 
      H_{m-1}^{n+1} (-\frac{\tau}{2h} (V_{m}^{n+1} + |V_{m}^{n+1}|))  + \\ \\
      \, + H_{m}^{n+1} (1 + \frac{\tau} {2h} (V_{m+1}^{n+1} + |V_{m+1}^{n+1}| - V_{m}^{n+1} + |V_{m}^{n+1}|)) +\\ \\
      \, + H_{m+1}^{n+1} (\frac{\tau}{2h} (V_{m+1}^{n+1} - |V_{m+1}^{n+1}|)) =\\ \\
      \, = \tau (f_{0})^{n}_{m} + H_{m}^{n},  \quad 0 \leq m<M, n \geq 0 \quad (4)
    \end{array}  
  \right .
$

\section{Отладочный тест}
\subsection{Постановка задачи}
Зададим функции
$$
\begin{aligned}
\tilde{\rho}(t, x) &=e^{t}(\cos (\frac{\pi x}{10})+1.5) \\
\tilde{u}(t, x) &=\cos (2 \pi t) \sin \left(\pi(\frac{x}{10})^{2}\right)
\end{aligned}
$$
Определим функции $f_{0}$ (отличную от нуля правую часть уравнения неразрывности) и $f$ следующим образом:
$$
\begin{array}{l}
\frac{\partial \tilde{\rho}}{\partial t}+\frac{\partial \tilde{\rho} \tilde{u}}{\partial x}=f_{0} \\
\tilde{\rho} \frac{\partial \tilde{u}}{\partial t}+\tilde{\rho} \tilde{u} \frac{\partial \tilde{u}}{\partial x}+\frac{\partial p}{\partial x}=\mu \frac{\partial^{2} \tilde{u}}{\partial x^{2}}+f \\
p=p(\tilde{\rho})
\end{array}
$$
Таким образом, дифференциальная задача для системы (*) с начальными и граничными условиями
$$
\begin{array}{l}
\tilde{\rho}(0, x)=\cos (\pi x / 10)+1.5, x \in[0,10] \\
\bar{u}(0, x)=\sin \left(\pi(x / 10)^{2}\right), x \in[0,10] \\
\tilde{u}(t, 0)=u(t, 10)=0, t \in[0,1]
\end{array}
$$
имеет гладкое точное решение в области $Q = [0,1] \times[0, 10]$, задаваемое функциями $\tilde{\rho}$
и $\tilde{u}$.
\\
Учитывая выражение для производной произведения:
$$
\frac{\partial \tilde{\rho}\tilde{u}}{\partial x} = 
\frac{\partial \tilde{\rho}}{\partial x}\tilde{u} + \frac{\partial \tilde{u}}{\partial x}\tilde{\rho}
$$
выпишем в явном виде функции, необходимые для подсчета функций $f$ и $f_{0}$:

$$
  \begin{array}{l}
    \tilde{\rho}_{t} = \tilde{\rho} \\
    \tilde{\rho}_{x} = -\frac{\pi}{10} e^{t} \sin (\frac{\pi x}{10}) \\
    \tilde{u}_{x} = \frac{\pi x}{50} \cos (2 \pi t) \cos ( \pi (\frac{x}      {10})^{2}) \\
    \tilde{u}_{t} = -2\pi \sin (2 \pi t) \sin (\pi (\frac{x}{10})^{2})\\
    \frac{\partial p}{\partial x} = \gamma \tilde{\rho}^{\gamma - 1}     \tilde{\rho}_{x} \\
    \frac{\partial^{2}\tilde{u}}{\partial x^{2}} = -\frac{\pi \cos (2 \pi t) (\pi x^{2} \sin (\frac{\pi x^{2}}{100}) - 50 \cos (\frac{\pi x^{2}}{100}))}{2500} 
    \end{array}
$$
\subsection{Численные эксперименты}
\subsubsection{Невязки в норме $C^{h}$}
\noindent

\begin{center}
  \begin{tabular}{ | l | l | l | l | l |}
    \hline
    \backslashbox{$\tau$}{$h$} & 0.1 & 0.01 &0.001 & 0.0001 \\ \hline
0.1 & 8.285230e-02 & 3.080613e+00 & 3.699816e+01 & 2.840256e+02 \\ \hline
0.01 & 3.293312e-03 & 1.714846e-03 & 1.638271e-03 & 1.632187e-03 \\ \hline
0.001 & 3.836806e-03 & 1.939109e-03 & 1.788066e-03 & 1.773883e-03 \\ \hline
0.0001 & 3.891207e-03 & 1.973442e-03 & 1.817514e-03 & 1.802784e-03 \\ \hline



  \end{tabular}
  $
  \text { Таблица 1: Ошибка решения для } V \text { при } \mu=10^{-1}$
\end{center}
\vfill
\begin{center}
  \begin{tabular}{ | l | l | l | l | l |}
    \hline
    \backslashbox{$\tau$}{$h$} & 0.1 & 0.01 &0.001 & 0.0001 \\ \hline
0.1 & 7.815377e-01 & 9.646356e+00 & 2.593327e+02 & 7.603305e+02 \\ \hline
0.01 & 2.201643e-03 & 2.038254e+00 & 2.228518e+01 & 2.744123e+03 \\ \hline
0.001 & 2.696648e-03 & 3.896222e-04 & 2.144925e-04 & 2.049316e-04 \\ \hline
0.0001 & 2.755748e-03 & 4.304076e-04 & 2.256746e-04 & 2.099311e-04 \\ \hline


  \end{tabular}
  $
  \text { Таблица 1: Ошибка решения для } V \text { при } \mu=10^{-2}$
\end{center}
\vfill
\begin{center}
  \begin{tabular}{ | l | l | l | l | l |}
    \hline
    \backslashbox{$\tau$}{$h$} & 0.1 & 0.01 &0.001 & 0.0001 \\ \hline
0.1 & 6.873624e-01 & 8.658355e+01 & 9.797400e+02 & 8.549658e+03 \\ \hline
0.01 & 2.098316e-03 & 5.667679e+00 & 7.078428e+01 & 5.507045e+02 \\ \hline
0.001 & 2.585017e-03 & 2.366792e-04 & 3.859310e+00 & 4.209609e+01 \\ \hline
0.0001 & 2.644665e-03 & 2.811675e-04 & 3.983411e-05 & 2.228885e-05 \\ \hline

  \end{tabular}
  $
  \text { Таблица 1: Ошибка решения для } V \text { при } \mu=10^{-3}$
\end{center}
\vfill
\begin{center}
  \begin{tabular}{ | l | l | l | l | l |}
    \hline
    \backslashbox{$\tau$}{$h$} & 0.1 & 0.01 &0.001 & 0.0001 \\ \hline
      0.1 & 8.346097e-01 & 1.006808e+01 & 2.041265e+01 & 7.610519e+00 \\ \hline
      0.01 & 5.311057e-02 & 4.851442e-02 & 4.812621e-02 & 4.808817e-02 \\ \hline
      0.001 & 1.204427e-02 & 5.370461e-03 & 5.139162e-03 & 5.124456e-03 \\ \hline
      0.0001 & 9.231614e-03 & 2.239485e-03 & 2.487148e-03 & 2.528856e-03 \\ \hline
  \end{tabular}
  $
  \text { Таблица 1: Ошибка решения для } H \text { при } \mu=10^{-1}$
\end{center}
\vfill
\begin{center}
  \begin{tabular}{ | l | l | l | l | l |}
    \hline
    \backslashbox{$\tau$}{$h$} & 0.1 & 0.01 &0.001 & 0.0001 \\ \hline
      0.1 & 2.824735e+00 & 2.344242e+01 & 1.027078e+02 & 4.043054e+02 \\ \hline
      0.01 & 5.360122e-02 & 2.228508e+01 & 6.714340e+02 & 7.169226e+02 \\ \hline
      0.001 & 1.294647e-02 & 5.311395e-03 & 4.840166e-03 & 4.800066e-03 \\ \hline
      0.0001 & 1.024173e-02 & 1.219743e-03 & 5.539551e-04 & 5.297192e-04 \\ \hline
  \end{tabular}
  $
  \text { Таблица 1: Ошибка решения для } H \text { при } \mu=10^{-2}$
\end{center}
\vfill
\begin{center}
  \begin{tabular}{ | l | l | l | l | l |}
    \hline 
      \backslashbox{$\tau$}{$h$} & 0.1 & 0.01 &0.001 & 0.0001 \\ \hline
      0.1 & 2.536448e+00 & 4.668277e+01 & 1.660260e+02 & 6.397873e+02 \\ \hline
      0.01 & 5.365330e-02 & 4.883634e+01 & 8.905963e+02 & 2.118093e+03 \\ \hline
      0.001 & 1.304692e-02 & 5.356476e-03 & 1.461903e+02 & 7.312164e+02 \\ \hline
      0.0001 & 1.035452e-02 & 1.296004e-03 & 5.311780e-04 & 4.839332e-04 \\ \hline

  \end{tabular}
  $
  \text { Таблица 1: Ошибка решения для } H \text { при } \mu=10^{-3}$
\end{center}
\vfill
\subsubsection{Невязки в норме $L_{2}^{h}$ }
\begin{center}
  \begin{tabular}{ | l | l | l | l | l |}
    \hline 
      \backslashbox{$\tau$}{$h$} & 0.1 & 0.01 &0.001 & 0.0001 \\ \hline
0.1 & 1.475580e-01 & 1.019468e+01 & 1.350498e+02 & 9.316197e+02 \\ \hline
0.01 & 3.028202e-03 & 1.884408e-03 & 1.767828e-03 & 1.756156e-03 \\ \hline
0.001 & 3.210045e-03 & 2.079068e-03 & 1.964051e-03 & 1.952539e-03 \\ \hline
0.0001 & 3.214100e-03 & 2.083451e-03 & 1.968507e-03 & 1.957003e-03 \\ \hline
\end{tabular}
  $ \text { Таблица 1: Ошибка решения для } V \text { при } \mu=10^{-1}$
\end{center}
\vfill
\begin{center}
  \begin{tabular}{ | l | l | l | l | l |}
    \hline 
      \backslashbox{$\tau$}{$h$} & 0.1 & 0.01 &0.001 & 0.0001 \\ \hline
0.1 & 1.140342e+00 & 2.735513e+01 & 1.611362e+03 & 1.578020e+04 \\ \hline
0.01 & 1.626927e-03 & 8.952561e+00 & 2.361110e+01 & 1.844710e+04 \\ \hline
0.001 & 1.879508e-03 & 4.303917e-04 & 2.906447e-04 & 2.768192e-04 \\ \hline
0.0001 & 1.891652e-03 & 4.413476e-04 & 3.015986e-04 & 2.877784e-04 \\ \hline
\end{tabular}
  $ \text { Таблица 1: Ошибка решения для } V \text { при } \mu=10^{-2}$
    \end{center}
\begin{center}
  \begin{tabular}{ | l | l | l | l | l |}
    \hline 
      \backslashbox{$\tau$}{$h$} & 0.1 & 0.01 &0.001 & 0.0001 \\ \hline
0.1 & 1.088811e+00 & 1.442646e+02 & 2.863084e+03 & 1.477274e+04 \\ \hline
0.01 & 1.484128e-03 & 3.892218e+01 & 1.860994e+02 & 6.915444e+03 \\ \hline
0.001 & 1.742462e-03 & 1.867017e-04 & 8.547429e+01 & 1.275444e+03 \\ \hline
0.0001 & 1.756639e-03 & 1.989882e-04 & 4.521465e-05 & 3.107465e-05 \\ \hline
\end{tabular}
  $ \text { Таблица 1: Ошибка решения для } V \text { при } \mu=10^{-3}$
    \end{center}
\vfill
\end{document}
