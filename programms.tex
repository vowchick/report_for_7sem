\documentclass[12pt]{article}
\usepackage{lipsum}

\makeatletter
\def\CTEX@section@format{\Large\bfseries}
\makeatother

\usepackage[T2A]{fontenc}
\usepackage[utf8]{inputenc}
\usepackage{diagbox}
\usepackage{amsmath}
\usepackage[english, russian]{babel}
\usepackage{listings}
\usepackage{graphicx}
\usepackage{tikz}
\usepackage{float}
\graphicspath{{pictures/}}
\DeclareGraphicsExtensions{.png}

\begin{document}
\begin{center} \large \bf
Мехмат. Сошников Владимир, группа 410.\\
Численное моделирование нестационарного одномерного течения газа с использованием неявной последовательной разностной схемы А.Г. Соколова   ($u$, $\rho$). 
\end{center}

\section{Постановка задачи}

В данной работе будет рассматриваться разностная схема А.Г. Соколова для решения начально - краевых задач для системы уравнений, описывающей нестационарное одномерное движение вязкого баротропного газа:

\begin{equation*}
 \begin{cases}
   \text
     {
       $
         \frac {\partial \rho} {\partial t} + 
         \frac {\partial \rho u} {\partial x} = 
         0
       $
     }
   \\
   \text 
     {
       $
          \rho \frac {\partial u} {\partial t} +
          \rho u \frac {\partial u} {\partial x} + 
          \frac {\partial p} {\partial x} =
          \mu \frac {\partial^{2} u} {\partial x ^{2}} +
          \rho f
       $
     }
   \\\tag{*}
   \text 
     {
       $
         p = p (\rho)
       $
     } 
   \\
   \text 
     {
       $
        (\rho, u)|_{t = 0} = (\rho_{0}, \,u_{0}), \,  x \in [0, X]
       $
     } 
   \\
   \text 
     {
       $
        u (t, 0) = u (t, X) = 0, \,t \in [0, T]
       $
     } 
 \end{cases}
\end{equation*}
Неизвестные функции: плотность $\rho$ и скорость $u$ являются функциями переменных Эйлера
$$
(t,x) \in Q = [0,T] \times [0,X].
$$
В уравнения входят еще две неизвестные функции - давление$p$, зависящее от плотности и вектор внешних сил $f$, являющийся функцией переменных Эйлера.
В данной работе рассматривается зависимость $p = \rho ^{\gamma}$, где $\gamma = 1.4$
\section{Описание алгоритма}
\subsection{Используемые обозначения}
Рассмотрим временной интервал $[0,T]$, введем на нем равномерную сетку:
$ \bar w_{\tau} = \{n \tau | n = 0,...,N\}$, $N \tau = T$
Рассмотрим пространственную область в виде отрезка $\bar \Omega = [0, X]$. Обозначим  через $\bar w_{h} = \{mh|m = 0,...,M\}$ (где $Mh = X$), равномерную сетку с шагом $h$, а
через $w_{h}^{1/2} = \{mh + \frac{h}{2} | m = 0,..., M - 1\}$ сдвинутую равномерную сетку с полуцелыми узлами.
Введем также обозначения для сеток:
$
\bar Q_{\tau h} := \bar \omega_{\tau} \times \bar \omega_{h}
$
и 
$
Q_{\tau h}^{1/2} = \bar \omega_{\tau} \times \omega_{h}^{1/2}
$

Значение функции $g$ в узле (m, n) будем обозначать через $g_{n}^{m}$
Для сокращения записей будем также использовать следующие обозначения:
\\
$
g_{m}^{n+1} = \hat g \
$,
$
g_{m \pm 1} ^{n} = g^{\pm 1}
$

Обозначения для среднего значения величин сеточной функции в двух соседних узлах: \\
$
g_{s} = \frac{g_{m+1}^{n} + g_{m}^{n}}{2}
$,
$
g_{\bar s} = \frac {g_{m}^{n} + g_{m-1}^{n}}{2}
$
\\
Для разностных операторов применяются следующие обозначения:
\\
$$
g_{t}=\frac{g_{m}^{n+1}-g_{m}^{n}}{\tau}, \quad g_{x}=\frac{g_{m+1}^{n}-g_{m}^{n}}{h},  \quad g_{\bar{x}}=\frac{g_{m}^{n}-g_{m-1}^{n}}{h}, \quad g_{x \bar{x}}=\frac{g_{m-1}^{n}-2 g_{m}^{n}+g_{m+1}^{n}}{h^{2}}
$$
Для приближения конвективных слагаемых в дифференциальных операторах будут использоваться разностные аппроксимации 
против потока. Для этих выражений введем обозначение:
$$
\delta\{W, V\}=\frac{V+|V|}{2} W_{\bar{x}}+\frac{V-|V|}{2} W_{x}=\left\{\begin{array}{l}
V W_{\bar{x}}, \text { если } V \geq 0 \\
V W_{x}, \text { если } V<0
\end{array}\right.
$$
В данной схеме в конвективных слагаемых узел шаблона, в котором нужно брать значение сеточной функции $F$,
 зависит от знака компоненты вектора скорости $V$. Для этих выражений используется обозначение:
$$
\sigma\{F, V\}=F \frac{|V|-V}{2|V|}+F^{(-1)} \frac{|V|+V}{2|V|}=\left\{\begin{array}{l}
F, \text { ecли } V<0 \\
F^{(-1)}, \text {если } V \geq 0
\end{array}\right.
$$

\subsection{Схема в веденных обозначениях}

    $ \left\{ \begin{array}{l}
H_{ \bar s} V_{t}+H_{\bar s} \delta\{\hat{V}, V\}+\frac{\gamma}{\gamma-1} H_{\bar{s}}\left((H)^{\gamma-1}\right)_{\bar x}=\mu \hat{V}_{x \bar{x}}+H_{\bar{s}} f, \quad \text { при } H_{\bar{s}} \neq 0 \quad (1)\\ \\
\hat{V}=0, \quad \text { при } H_{\bar{s}}=0 \quad (2) \\ \\
0<m<M, n \geq 0 \\ \\
\hat{V}_{0}=\hat{V}_{M}=0, \quad (3)\\ \\
H_{t}+(\sigma\{\hat{H}, \hat{V}\} \hat{V})_{x}=0,  \quad 0 \leq m<M, n \geq 0 \quad (4)
\end{array}
\right.
$

\subsection{Поточечная запись схемы}
$
  \left \{
    \begin{array}{l}
      V_{m-1}^{n+1} (-\frac{\tau}{4h} (V_{m}^{n} + |V_{m}^{n}|) (H_{m}^{n} + H_{m - 1} ^{n}) - \frac {\tau \mu} {h^{2}}) \, + \\ \\ 
      + \,V_{m}^{n + 1} (\frac{H_{m}^{n} + H_{m - 1} ^ {n}}{2} + \frac{2 \tau \mu} {h^{2}} + \frac{\tau}{2h} |V_{m}^{n}|(H_{m}^{n} + H_{m -  1} ^ {n})) +\\ \\
      + \,V_{m+1}^{n+1} (\frac {\tau}{4h} (V_{m}^{n} - |V_{m}^{n}|) (H_{m}^{n} + H_{m-1}^{n}) - \frac{\tau \mu} {h^{2}})  = \\   \\
      =\, V_{m}^{n} \frac{H_{m}^{n} + H_{m - 1}^{n}} {2} - \frac {\tau \gamma}{2h (\gamma - 1)} (H_{m}^{n} + H_{m - 1}^{n}) ((H_{m}^{n})^{\gamma - 1} - 
      (H_{m - 1}^{n})^{\gamma - 1}) +  \\ \\ 
      + \,\tau f_{m}^{n} \frac{H_{m}^{n} + H_{m-1}^{n}}{2}, \quad \text{при} \, \frac{H_{m}^{n} + H_{m - 1}^{n}}{2} \neq 0 \quad (1) \\ \\
      0<m<M, n \geq 0 \\ \\
      V_{m}^{n+1} = 0, \quad \text{при} \, \frac{H_{m}^{n} + H_{m - 1}^{n}}{2} = 0 \quad (2)\\ \\
      V_{0}^{n+1} = V_{M}^{n+1} = 0 \quad(3) \\ \\ 
      H_{m-1}^{n+1} (-\frac{\tau}{2h} (V_{m}^{n+1} + |V_{m}^{n+1}|))  + \\ \\
      \, + H_{m}^{n+1} (1 + \frac{\tau} {2h} (V_{m+1}^{n+1} + |V_{m+1}^{n+1}| - V_{m}^{n+1} + |V_{m}^{n+1}|)) +\\ \\
      \, + H_{m+1}^{n+1} (\frac{\tau}{2h} (V_{m+1}^{n+1} - |V_{m+1}^{n+1}|)) =\\ \\
      \, = \tau (f_{0})^{n}_{m} + H_{m}^{n},  \quad 0 \leq m<M, n \geq 0 \quad (4)
    \end{array}  
  \right .
$

\section{Отладочный тест}
\subsection{Постановка задачи}
Зададим функции
$$
\begin{aligned}
\tilde{\rho}(t, x) &=e^{t}(\cos (\frac{\pi x}{10})+1.5) \\
\tilde{u}(t, x) &=\cos (2 \pi t) \sin \left(\pi(\frac{x}{10})^{2}\right)
\end{aligned}
$$
Определим функции $f_{0}$ (отличную от нуля правую часть уравнения неразрывности) и $f$ следующим образом:
$$
\begin{array}{l}
\frac{\partial \tilde{\rho}}{\partial t}+\frac{\partial \tilde{\rho} \tilde{u}}{\partial x}=f_{0} \\
\tilde{\rho} \frac{\partial \tilde{u}}{\partial t}+\tilde{\rho} \tilde{u} \frac{\partial \tilde{u}}{\partial x}+\frac{\partial p}{\partial x}=\mu \frac{\partial^{2} \tilde{u}}{\partial x^{2}}+f \\
p=p(\tilde{\rho})
\end{array}
$$
Таким образом, дифференциальная задача для системы (*) с начальными и граничными условиями
$$
\begin{array}{l}
\tilde{\rho}(0, x)=\cos (\pi x / 10)+1.5, x \in[0,10] \\
\bar{u}(0, x)=\sin \left(\pi(x / 10)^{2}\right), x \in[0,10] \\
\tilde{u}(t, 0)=u(t, 10)=0, t \in[0,1]
\end{array}
$$
имеет гладкое точное решение в области $Q = [0,1] \times[0, 10]$, задаваемое функциями $\tilde{\rho}$
и $\tilde{u}$.
\\
Учитывая выражение для производной произведения:
$$
\frac{\partial \tilde{\rho}\tilde{u}}{\partial x} = 
\frac{\partial \tilde{\rho}}{\partial x}\tilde{u} + \frac{\partial \tilde{u}}{\partial x}\tilde{\rho}
$$
выпишем в явном виде функции, необходимые для подсчета функций $f$ и $f_{0}$:

$$
  \begin{array}{l}
    \tilde{\rho}_{t} = \tilde{\rho} \\
    \tilde{\rho}_{x} = -\frac{\pi}{10} e^{t} \sin (\frac{\pi x}{10}) \\
    \tilde{u}_{x} = \frac{\pi x}{50} \cos (2 \pi t) \cos ( \pi (\frac{x}      {10})^{2}) \\
    \tilde{u}_{t} = -2\pi \sin (2 \pi t) \sin (\pi (\frac{x}{10})^{2})\\
    \frac{\partial p}{\partial x} = \gamma \tilde{\rho}^{\gamma - 1}     \tilde{\rho}_{x} \\
    \frac{\partial^{2}\tilde{u}}{\partial x^{2}} = -\frac{\pi \cos (2 \pi t) (\pi x^{2} \sin (\frac{\pi x^{2}}{100}) - 50 \cos (\frac{\pi x^{2}}{100}))}{2500} 
    \end{array}
$$
\subsection{Численные эксперименты}
\subsubsection{Невязки в норме $C^{h}$}
\noindent

\begin{center}
  \begin{tabular}{ | l | l | l | l | l |}
    \hline
    \backslashbox{$\tau$}{$h$} & 0.1 & 0.01 &0.001 & 0.0001 \\ \hline
0.1 & 8.285230e-02 & 3.080613e+00 & 3.699816e+01 & 2.840256e+02 \\ \hline
0.01 & 3.293312e-03 & 1.714846e-03 & 1.638271e-03 & 1.632187e-03 \\ \hline
0.001 & 3.836806e-03 & 1.939109e-03 & 1.788066e-03 & 1.773883e-03 \\ \hline
0.0001 & 3.891207e-03 & 1.973442e-03 & 1.817514e-03 & 1.802784e-03 \\ \hline



  \end{tabular}
  $
  \text { Таблица: Ошибка решения для } V \text { при } \mu=10^{-1}$
\end{center}
\vfill
\begin{center}
  \begin{tabular}{ | l | l | l | l | l |}
    \hline
    \backslashbox{$\tau$}{$h$} & 0.1 & 0.01 &0.001 & 0.0001 \\ \hline
0.1 & 7.815377e-01 & 9.646356e+00 & 2.593327e+02 & 7.603305e+02 \\ \hline
0.01 & 2.201643e-03 & 2.038254e+00 & 2.228518e+01 & 2.744123e+03 \\ \hline
0.001 & 2.696648e-03 & 3.896222e-04 & 2.144925e-04 & 2.049316e-04 \\ \hline
0.0001 & 2.755748e-03 & 4.304076e-04 & 2.256746e-04 & 2.099311e-04 \\ \hline


  \end{tabular}
  $
  \text { Таблица: Ошибка решения для } V \text { при } \mu=10^{-2}$
\end{center}
\vfill
\begin{center}
  \begin{tabular}{ | l | l | l | l | l |}
    \hline
    \backslashbox{$\tau$}{$h$} & 0.1 & 0.01 &0.001 & 0.0001 \\ \hline
0.1 & 6.873624e-01 & 8.658355e+01 & 9.797400e+02 & 8.549658e+03 \\ \hline
0.01 & 2.098316e-03 & 5.667679e+00 & 7.078428e+01 & 5.507045e+02 \\ \hline
0.001 & 2.585017e-03 & 2.366792e-04 & 3.859310e+00 & 4.209609e+01 \\ \hline
0.0001 & 2.644665e-03 & 2.811675e-04 & 3.983411e-05 & 2.228885e-05 \\ \hline

  \end{tabular}
  $
  \text { Таблица: Ошибка решения для } V \text { при } \mu=10^{-3}$
\end{center}
\vfill
\begin{center}
  \begin{tabular}{ | l | l | l | l | l |}
    \hline
    \backslashbox{$\tau$}{$h$} & 0.1 & 0.01 &0.001 & 0.0001 \\ \hline
      0.1 & 8.346097e-01 & 1.006808e+01 & 2.041265e+01 & 7.610519e+00 \\ \hline
      0.01 & 5.311057e-02 & 4.851442e-02 & 4.812621e-02 & 4.808817e-02 \\ \hline
      0.001 & 1.204427e-02 & 5.370461e-03 & 5.139162e-03 & 5.124456e-03 \\ \hline
      0.0001 & 9.231614e-03 & 2.239485e-03 & 2.487148e-03 & 2.528856e-03 \\ \hline
  \end{tabular}
  $
  \text { Таблица: Ошибка решения для } H \text { при } \mu=10^{-1}$
\end{center}
\vfill
\begin{center}
  \begin{tabular}{ | l | l | l | l | l |}
    \hline
    \backslashbox{$\tau$}{$h$} & 0.1 & 0.01 &0.001 & 0.0001 \\ \hline
      0.1 & 2.824735e+00 & 2.344242e+01 & 1.027078e+02 & 4.043054e+02 \\ \hline
      0.01 & 5.360122e-02 & 2.228508e+01 & 6.714340e+02 & 7.169226e+02 \\ \hline
      0.001 & 1.294647e-02 & 5.311395e-03 & 4.840166e-03 & 4.800066e-03 \\ \hline
      0.0001 & 1.024173e-02 & 1.219743e-03 & 5.539551e-04 & 5.297192e-04 \\ \hline
  \end{tabular}
  $
  \text { Таблица: Ошибка решения для } H \text { при } \mu=10^{-2}$
\end{center}
\vfill
\begin{center}
  \begin{tabular}{ | l | l | l | l | l |}
    \hline 
      \backslashbox{$\tau$}{$h$} & 0.1 & 0.01 &0.001 & 0.0001 \\ \hline
      0.1 & 2.536448e+00 & 4.668277e+01 & 1.660260e+02 & 6.397873e+02 \\ \hline
      0.01 & 5.365330e-02 & 4.883634e+01 & 8.905963e+02 & 2.118093e+03 \\ \hline
      0.001 & 1.304692e-02 & 5.356476e-03 & 1.461903e+02 & 7.312164e+02 \\ \hline
      0.0001 & 1.035452e-02 & 1.296004e-03 & 5.311780e-04 & 4.839332e-04 \\ \hline

  \end{tabular}
  $
  \text { Таблица: Ошибка решения для } H \text { при } \mu=10^{-3}$
\end{center}
\vfill
\subsubsection{Невязки в норме $L_{2}^{h}$ }
\begin{center}
  \begin{tabular}{ | l | l | l | l | l |}
    \hline 
      \backslashbox{$\tau$}{$h$} & 0.1 & 0.01 &0.001 & 0.0001 \\ \hline
0.1 & 1.475580e-01 & 1.019468e+01 & 1.350498e+02 & 9.316197e+02 \\ \hline
0.01 & 3.028202e-03 & 1.884408e-03 & 1.767828e-03 & 1.756156e-03 \\ \hline
0.001 & 3.210045e-03 & 2.079068e-03 & 1.964051e-03 & 1.952539e-03 \\ \hline
0.0001 & 3.214100e-03 & 2.083451e-03 & 1.968507e-03 & 1.957003e-03 \\ \hline
\end{tabular}
  $ \text { Таблица: Ошибка решения для } V \text { при } \mu=10^{-1}$
\end{center}
\vfill
\begin{center}
  \begin{tabular}{ | l | l | l | l | l |}
    \hline 
      \backslashbox{$\tau$}{$h$} & 0.1 & 0.01 &0.001 & 0.0001 \\ \hline
0.1 & 1.140342e+00 & 2.735513e+01 & 1.611362e+03 & 1.578020e+04 \\ \hline
0.01 & 1.626927e-03 & 8.952561e+00 & 2.361110e+01 & 1.844710e+04 \\ \hline
0.001 & 1.879508e-03 & 4.303917e-04 & 2.906447e-04 & 2.768192e-04 \\ \hline
0.0001 & 1.891652e-03 & 4.413476e-04 & 3.015986e-04 & 2.877784e-04 \\ \hline
\end{tabular}
  $ \text { Таблица: Ошибка решения для } V \text { при } \mu=10^{-2}$
    \end{center}
\begin{center}
  \begin{tabular}{ | l | l | l | l | l |}
    \hline 
      \backslashbox{$\tau$}{$h$} & 0.1 & 0.01 &0.001 & 0.0001 \\ \hline
0.1 & 1.088811e+00 & 1.442646e+02 & 2.863084e+03 & 1.477274e+04 \\ \hline
0.01 & 1.484128e-03 & 3.892218e+01 & 1.860994e+02 & 6.915444e+03 \\ \hline
0.001 & 1.742462e-03 & 1.867017e-04 & 8.547429e+01 & 1.275444e+03 \\ \hline
0.0001 & 1.756639e-03 & 1.989882e-04 & 4.521465e-05 & 3.107465e-05 \\ \hline
\end{tabular}
  $ \text { Таблица: Ошибка решения для } V \text { при } \mu=10^{-3}$
    \end{center}
\vfill
\begin{center}
  \begin{tabular}{ | l | l | l | l | l |}
    \hline 
      \backslashbox{$\tau$}{$h$} & 0.1 & 0.01 &0.001 & 0.0001 \\ \hline
0.1 & 1.293417e+00 & 7.833805e+01 & 5.146109e+02 & 1.950443e+01 \\ \hline
0.01 & 2.518932e-02 & 2.431426e-02 & 2.422312e-02 & 2.421398e-02 \\ \hline
0.001 & 5.275670e-03 & 4.033627e-03 & 3.946299e-03 & 3.937546e-03 \\ \hline
0.0001 & 4.676362e-03 & 2.016298e-03 & 2.070623e-03 & 2.116467e-03 \\ \hline
   \end{tabular}
  $ \text { Таблица: Ошибка решения для } H \text { при } \mu=10^{-1}$
\end{center}
\vfill
\begin{center}
  \begin{tabular}{ | l | l | l | l | l |}
    \hline 
      \backslashbox{$\tau$}{$h$} & 0.1 & 0.01 &0.001 & 0.0001 \\ \hline
0.1 & 3.841700e+00 & 1.865971e+02 & 3.035746e+03 & 3.081776e+04 \\ \hline
0.01 & 2.461036e-02 & 1.963539e+02 & 2.120503e+04 & 7.001986e+04 \\ \hline
0.001 & 5.661716e-03 & 2.541027e-03 & 2.447665e-03 & 2.438426e-03 \\ \hline
0.0001 & 5.529478e-03 & 5.292462e-04 & 4.262702e-04 & 4.172832e-04 \\ \hline
  \end{tabular}
  $ \text { Таблица: Ошибка решения для } H \text { при } \mu=10^{-2}$
 \end{center}
\vfill
\begin{center}
  \begin{tabular}{ | l | l | l | l | l |}
    \hline 
      \backslashbox{$\tau$}{$h$} & 0.1 & 0.01 &0.001 & 0.0001 \\ \hline
0.1 & 3.894058e+00 & 3.512253e+02 & 3.963516e+03 & 3.977753e+04 \\ \hline
0.01 & 2.462919e-02 & 3.835050e+02 & 2.815793e+04 & 2.078885e+05 \\ \hline
0.001 & 5.699719e-03 & 2.444318e-03 & 3.156271e+03 & 7.007184e+04 \\ \hline
0.0001 & 5.588382e-03 & 5.654515e-04 & 2.543140e-04 & 2.449570e-04 \\ \hline
\end{tabular}
  $ \text { Таблица: Ошибка решения для } H \text { при } \mu=10^{-3}$
    \end{center}
\vfill
\subsubsection{Невязки в норме $W_{2}^{1, h}$ }
\begin{center}
  \begin{tabular}{ | l | l | l | l | l |}
    \hline 
      \backslashbox{$\tau$}{$h$} & 0.1 & 0.01 &0.001 & 0.0001 \\ \hline
0.1 & 4.383271e-01 & 6.731418e+01 & 2.667770e+03 & 1.986627e+04 \\ \hline
0.01 & 6.838589e-03 & 5.511401e-03 & 5.548671e-03 & 5.556697e-03 \\ \hline
0.001 & 7.290102e-03 & 5.486723e-03 & 5.472968e-03 & 5.474921e-03 \\ \hline
0.0001 & 7.327100e-03 & 5.496580e-03 & 5.475200e-03 & 5.473536e-03 \\ \hline
  \end{tabular}
  $ \text { Таблица: Ошибка решения для } V \text { при } \mu=10^{-1}$
\end{center}
\vfill
\begin{center}
  \begin{tabular}{ | l | l | l | l | l |}
    \hline 
      \backslashbox{$\tau$}{$h$} & 0.1 & 0.01 &0.001 & 0.0001 \\ \hline
0.1 & 4.310467e+00 & 5.460371e+02 & 5.333656e+04 & 1.442581e+06 \\ \hline
0.01 & 3.951909e-03 & 1.353896e+02 & 8.557853e+02 & 6.901876e+05 \\ \hline
0.001 & 4.381496e-03 & 9.930228e-04 & 8.457272e-04 & 8.420778e-04 \\ \hline
0.0001 & 4.432791e-03 & 1.004503e-03 & 8.278189e-04 & 8.209564e-04 \\ \hline
  \end{tabular}
  $ \text { Таблица: Ошибка решения для } V \text { при } \mu=10^{-2}$
\end{center}
\vfill

\begin{center}
  \begin{tabular}{ | l | l | l | l | l |}
    \hline 
      \backslashbox{$\tau$}{$h$} & 0.1 & 0.01 &0.001 & 0.0001 \\ \hline
0.1 & 3.848226e+00 & 2.541506e+03 & 1.476536e+05 & 9.849140e+06 \\ \hline
0.01 & 3.652670e-03 & 5.481894e+02 & 1.185947e+04 & 1.521601e+06 \\ \hline
0.001 & 4.134590e-03 & 4.484777e-04 & 2.939740e+03 & 1.571153e+05 \\ \hline
0.0001 & 4.188599e-03 & 4.872382e-04 & 1.074311e-04 & 9.301182e-05 \\ \hline
  \end{tabular}
  $ \text { Таблица: Ошибка решения для } V \text { при } \mu=10^{-3}$
\end{center}
\vfill

\begin{center}
  \begin{tabular}{ | l | l | l | l | l |}
    \hline 
      \backslashbox{$\tau$}{$h$} & 0.1 & 0.01 &0.001 & 0.0001 \\ \hline
0.1 & 4.468332e+00 & 1.210453e+03 & 1.717035e+04 & 8.995577e+01 \\ \hline
0.01 & 5.511195e-02 & 5.003127e-02 & 4.962588e-02 & 4.958614e-02 \\ \hline
0.001 & 1.446086e-02 & 8.989757e-03 & 8.975376e-03 & 8.982159e-03 \\ \hline
0.0001 & 1.182303e-02 & 7.393480e-03 & 7.657063e-03 & 7.693473e-03 \\ \hline
  \end{tabular}
  $ \text { Таблица: Ошибка решения для } H \text { при } \mu=10^{-1}$
\end{center}
\vfill
\begin{center}
  \begin{tabular}{ | l | l | l | l | l |}
    \hline 
      \backslashbox{$\tau$}{$h$} & 0.1 & 0.01 &0.001 & 0.0001 \\ \hline
0.1 & 1.149008e+01 & 3.015928e+03 & 1.422972e+05 & 6.539492e+06 \\ \hline
0.01 & 5.532614e-02 & 2.220540e+03 & 6.131688e+05 & 7.500146e+06 \\ \hline
0.001 & 1.442992e-02 & 5.570949e-03 & 5.049295e-03 & 5.006587e-03 \\ \hline
0.0001 & 1.178786e-02 & 1.666307e-03 & 1.174305e-03 & 1.167886e-03 \\ \hline
  \end{tabular}
  $ \text { Таблица: Ошибка решения для } H \text { при } \mu=10^{-2}$
\end{center}
\vfill
\begin{center}
  \begin{tabular}{ | l | l | l | l | l |}
    \hline 
      \backslashbox{$\tau$}{$h$} & 0.1 & 0.01 &0.001 & 0.0001 \\ \hline
0.1 & 1.228315e+01 & 6.253254e+03 & 2.325642e+05 & 1.299162e+07 \\ \hline
0.01 & 5.537335e-02 & 4.810062e+03 & 1.262087e+06 & 4.093729e+07 \\ \hline
0.001 & 1.455101e-02 & 5.525931e-03 & 7.085203e+04 & 1.109121e+07 \\ \hline
0.0001 & 1.192958e-02 & 1.449296e-03 & 5.591742e-04 & 5.069650e-04 \\ \hline
  \end{tabular}
  $ \text { Таблица: Ошибка решения для } H \text { при } \mu=10^{-3}$
\end{center}
\vfill
\subsection{Выводы}
В итоге, можно сделать вывод, что система сходится в зависимости от параметра $\mu$, при $\tau < C h$, причем $C$ уменьшается с уменьшением $\mu$. Сходимость имеет порядок $\tau +  h$.
\section{Тесты с негладкими начальными данными}
\subsection{Постановка задачи}
Пусть $\Omega_{x}=[0 ; 10]$. Для системы (*) зададим две задачи, начальные и граничные условия которых определяются следующим образом:
$$
\begin{array}{l}
\rho(0, x)=\left\{\begin{array}{ll}
2, & \text { если } x \in[4.5 ; 5.5] \\
1, & \text { иначе; }
\end{array}\right. \\
u(0, x)=0 \\
u(t, 0)=u(t, 10)=0
\end{array}
$$
и
$$
\begin{array}{l}
\rho(0, x)=1 ; \\
u(0, x)=\left\{\begin{array}{ll}
1, & \text { ести } x \in[4.5 ; 5.5] ; \\
0, & \text { иначе; }
\end{array}\right. \\
u(t, 0)=u(t, 10)=0
\end{array}
$$
Положим также $f \equiv 0$ и $f_{0} \equiv 0 .$
Суть эксперимента состоит в решении этих задач, причем вычисления следует проводить до момента времени $T=N_{0} \tau,$ для которого
$$
\left\|V^{N_{0}}\right\|=\max _{0 \leq m \leq M}\left|V_{m}^{N_{0}}\right| \leq \varepsilon
$$
где величина $\varepsilon$ является достаточно малой и определяется опытным путем. Кроме того, для проверки консервативности системы определим массу газа на шаге $n$
$$
m(n)=h \sum_{m=0}^{M} H_{m}^{n}
$$
и введем функцию
$$
\Delta_{m}(n)=\frac{m(n)-m(0)}{m(0)}
$$
\subsection{Численные эксперименты для первой задачи}
\subsubsection{Точность решения}
Фиксируем $\varepsilon = 10^{-3}$. Далее приведены таблицы значений $\left\|V^{n}\right\|$
для $n = N_{0} / 4, N_{0} / 2, 3N_{0} / 4, N_{0}$ 

\begin{center}
  \begin{tabular}{| l | l | l | l | l | l | }
    \hline
        $\tau \times h$ & $N_{0}\tau$ & $N_{0} / 4$ & $N_{0} / 2$ & $3N_{0} / 4$ & $N_{0}$  \\ \hline
$0.0002 \times 0.01$ & 201.489 & 5.869542e-02 & 1.456946e-02 & 1.795449e-02 & 9.995019e-04  \\ \hline
$0.0002 \times 0.005$ & 201.495 & 5.888486e-02 & 1.467960e-02 & 1.803345e-02 & 9.993725e-04  \\ \hline
$0.0001 \times 0.01$ & 201.489 & 5.869644e-02 & 1.456269e-02 & 1.795364e-02 & 9.996790e-04  \\ \hline
$0.0001 \times 0.005$ & 201.495 & 5.888585e-02 & 1.467270e-02 & 1.803257e-02 & 9.995263e-04  \\ \hline
  \end{tabular}
  $ \text {Нормы скорости при } \mu=10^{-1}$
\end{center}
\vfill

\begin{center}
  \begin{tabular}{| l | l | l | l | l | l | }
    \hline
        $\tau \times h$ & $N_{0}\tau$ & $N_{0} / 4$ & $N_{0} / 2$ & $3N_{0} / 4$ & $N_{0}$  \\ \hline
$0.0002 \times 0.01$ & 1153.54 & 1.642106e-02 & 6.470636e-03 & 5.628159e-03 & 9.996306e-04  \\ \hline
$0.0002 \times 0.005$ & 1157.27 & 2.337882e-02 & 8.317921e-03 & 3.369622e-03 & 9.996882e-04  \\ \hline
$0.0001 \times 0.01$ & 1153.53 & 1.643073e-02 & 6.466305e-03 & 5.629139e-03 & 9.998170e-04  \\ \hline
$0.0001 \times 0.005$ & 1157.27 & 2.338543e-02 & 8.321749e-03 & 3.371166e-03 & 9.999187e-04  \\ \hline
  \end{tabular}
  $ \text {Нормы скорости при } \mu=10^{-2}$
\end{center}
\vfill

\begin{center}
  \begin{tabular}{| l | l | l | l | l | l | }
    \hline
        $\tau \times h$ & $N_{0}\tau$ & $N_{0} / 4$ & $N_{0} / 2$ & $3N_{0} / 4$ & $N_{0}$  \\ \hline
$0.0002 \times 0.01$ & 2941.43 & 9.899381e-03 & 3.939488e-03 & 1.976674e-03 & 9.998926e-04  \\ \hline
$0.0002 \times 0.005$ & 2967.38 & 7.186613e-03 & 3.395565e-03 & 2.912057e-03 & 9.998657e-04  \\ \hline
$0.0001 \times 0.01$ & 2941.42 & 9.925506e-03 & 3.949223e-03 & 1.980073e-03 & 9.999280e-04  \\ \hline
$0.0001 \times 0.005$ & 2967.37 & 7.219114e-03 & 3.384289e-03 & 2.916229e-03 & 9.999620e-04  \\ \hline
  \end{tabular}
  $ \text {Нормы скорости при } \mu=10^{-3}$
\end{center}
\vfill

\subsubsection{Консервативность системы}
\begin{center}
  \begin{tabular}{| l | l | l | l | l | l |}     \hline
      $\tau \times h$ & $N_{0} / 5$ & $2N_{0} / 5$ & $3N_{0} / 5$ & $4N_{0}/5$ & $N_{0}$  \\ \hline
$0.0002 \times 0.01$ & -9.349247e-16 & 1.495879e-15 & 1.682864e-15 & 1.869849e-16 & -4.113668e-15  \\ \hline
$0.0002 \times 0.005$ & 4.487638e-15 & 1.869849e-15 & 3.739699e-15 & 2.056834e-15 & 1.682864e-15  \\ \hline
$0.0001 \times 0.01$ & 2.804774e-15 & 6.731457e-15 & 3.926684e-15 & -2.056834e-15 & -3.926684e-15  \\ \hline
$0.0001 \times 0.005$ & 5.048593e-15 & 6.544473e-15 & 1.009719e-14 & 6.731457e-15 & 1.495879e-15  \\ \hline
  \end{tabular}
  $ \text {Разность масс  при } \mu=10^{-1}$
\end{center}
\vfill

\begin{center}
  \begin{tabular}{| l | l | l | l | l | l |}     \hline
      $\tau \times h$ & $N_{0} / 5$ & $2N_{0} / 5$ & $3N_{0} / 5$ & $4N_{0}/5$ & $N_{0}$  \\ \hline
$0.0002 \times 0.01$ & -2.243819e-15 & 4.113668e-15 & 3.365729e-15 & 5.422563e-15 & 7.292412e-15  \\ \hline
$0.0002 \times 0.005$ & -3.178744e-15 & -5.609548e-15 & -7.479397e-15 & -6.357488e-15 & -6.918442e-15  \\ \hline
$0.0001 \times 0.01$ & 1.084513e-14 & 2.729980e-14 & 4.394146e-14 & 3.982779e-14 & 3.683603e-14  \\ \hline
$0.0001 \times 0.005$ & 9.349247e-16 & 1.308895e-15 & -7.479397e-16 & -4.861608e-15 & -1.589372e-14  \\ \hline

  \end{tabular}
  $ \text {Разность масс при } \mu=10^{-2}$
\end{center}
\vfill

\subsubsection{Графики}
\begin{figure}[H]
\center{\includegraphics[scale=1]{a01.txt.png}}
\caption{График при $\mu=10^{-1}$}
\label{fig:image}
\end{figure}

\begin{figure}[H]
\center{\includegraphics[scale=1]{a001.txt.png}}
\caption{График при $\mu=10^{-2}$}
\label{fig:image}
\end{figure}

\begin{figure}[H]
\center{\includegraphics[scale=1]{a0001.txt.png}}
\caption{График при $\mu=10^{-3}$}
\label{fig:image}
\end{figure}

\subsection{Численные эксперименты для второй задачи}
\subsubsection{Точность решения}
\begin{center}
  \begin{tabular}{| l | l | l | l | l | l | }
    \hline
        $\tau \times h$ & $N_{0}\tau$ & $N_{0} / 4$ & $N_{0} / 2$ & $3N_{0} / 4$ & $N_{0}$  \\ \hline
$0.0002 \times 0.01$ & 392.392 & 4.073000e-02 & 1.942063e-02 & 1.616920e-02 & 9.997637e-04  \\ \hline
$0.0002 \times 0.005$ & 392.383 & 4.091964e-02 & 1.941650e-02 & 1.620574e-02 & 9.997746e-04  \\ \hline
$0.0001 \times 0.01$ & 392.391 & 4.074091e-02 & 1.941634e-02 & 1.617060e-02 & 9.999157e-04  \\ \hline
$0.0001 \times 0.005$ & 392.382 & 4.093176e-02 & 1.941125e-02 & 1.620723e-02 & 9.998260e-04  \\ \hline
  \end{tabular}
  $ \text {Нормы скорости при } \mu=10^{-1}$
\end{center}
\vfill

\begin{center}
  \begin{tabular}{| l | l | l | l | l | l | }
    \hline
        $\tau \times h$ & $N_{0}\tau$ & $N_{0} / 4$ & $N_{0} / 2$ & $3N_{0} / 4$ & $N_{0}$  \\ \hline
$0.0002 \times 0.01$ & 2173.46 & 9.514794e-03 & 8.210254e-03 & 4.411417e-03 & 9.998720e-04  \\ \hline
$0.0002 \times 0.005$ & 2207.16 & 9.840095e-03 & 8.382343e-03 & 4.394345e-03 & 9.999142e-04  \\ \hline
$0.0001 \times 0.01$ & 2181.89 & 1.400346e-02 & 4.034190e-03 & 5.546189e-03 & 9.999979e-04  \\ \hline
$0.0001 \times 0.005$ & 2215.58 & 1.441699e-02 & 4.045814e-03 & 5.577639e-03 & 9.999281e-04  \\ \hline
  \end{tabular}
  $ \text {Нормы скорости при } \mu=10^{-2}$
\end{center}
\vfill

\begin{center}
  \begin{tabular}{| l | l | l | l | l | l | }
    \hline
        $\tau \times h$ & $N_{0}\tau$ & $N_{0} / 4$ & $N_{0} / 2$ & $3N_{0} / 4$ & $N_{0}$  \\ \hline
$0.0002 \times 0.01$ & 5933.73 & 7.925314e-03 & 2.432744e-03 & 3.136509e-03 & 9.999799e-04  \\ \hline
$0.0002 \times 0.005$ & 6009.62 & 7.504520e-03 & 4.790401e-03 & 2.272567e-03 & 9.999933e-04  \\ \hline
$0.0001 \times 0.01$ & 5950.6 & 4.268523e-03 & 2.453016e-03 & 1.495137e-03 & 9.999809e-04  \\ \hline
$0.0001 \times 0.005$ & 6018.04 & 4.993029e-03 & 2.457929e-03 & 1.485679e-03 & 9.999975e-04  \\ \hline
  \end{tabular}
  $ \text {Нормы скорости при } \mu=10^{-3}$
\end{center}
\vfill

\subsubsection{Консервативность системы}
\begin{center}
  \begin{tabular}{| l | l | l | l | l | }
    \hline
	$\tau \times h$ & $N_{0} / 4$ & $N_{0} / 2$ & $3N_{0} / 4$ & $N_{0}$  \\ \hline
$0.0002 \times 0.01$ & -3.019807e-15 & -2.309264e-15 & -2.486900e-15 & -1.421085e-15  \\ \hline
$0.0002 \times 0.005$ & -7.638334e-15 & -8.348877e-15 & -7.460699e-15 & -2.842171e-15  \\ \hline
$0.0001 \times 0.01$ & -6.927792e-15 & -4.263256e-15 & -5.861978e-15 & -4.085621e-15  \\ \hline
$0.0001 \times 0.005$ & 7.105427e-16 & 1.421085e-15 & -8.881784e-16 & -1.776357e-15  \\ \hline
  \end{tabular}
  $ \text {Разность масс при  } \mu=10^{-1}$
\end{center}
\vfill

\begin{center}
  \begin{tabular}{| l | l | l | l | l | }
    \hline
	$\tau \times h$ & $N_{0} / 4$ & $N_{0} / 2$ & $3N_{0} / 4$ & $N_{0}$  \\ \hline
$0.0002 \times 0.01$ & -4.440892e-15 & 7.105427e-16 & -1.776357e-15 & -3.552714e-16  \\ \hline
$0.0002 \times 0.005$ & -4.973799e-15 & -7.815970e-15 & -3.197442e-15 & -4.263256e-15  \\ \hline
$0.0001 \times 0.01$ & -3.552714e-14 & -2.913225e-14 & -2.362555e-14 & -1.314504e-14  \\ \hline
$0.0001 \times 0.005$ & -2.575717e-14 & -2.131628e-14 & -2.113865e-14 & -1.225686e-14  \\ \hline
  \end{tabular}
  $ \text {Разность масс при  } \mu=10^{-2}$
\end{center}
\vfill

\begin{center}
  \begin{tabular}{| l | l | l | l | l | }
    \hline
	$\tau \times h$ & $N_{0} / 4$ & $N_{0} / 2$ & $3N_{0} / 4$ & $N_{0}$  \\ \hline
$0.0002 \times 0.01$ & -1.456613e-14 & -1.723066e-14 & -4.973799e-15 & -3.375078e-15  \\ \hline
$0.0002 \times 0.005$ & 4.796163e-15 & -6.927792e-15 & -8.704149e-15 & -8.526513e-15  \\ \hline
$0.0001 \times 0.01$ & 1.776357e-14 & 1.207923e-14 & 6.039613e-15 & -4.085621e-15  \\ \hline
$0.0001 \times 0.005$ & -9.059420e-15 & -1.474376e-14 & -7.815970e-15 & -1.776357e-15  \\ \hline
  \end{tabular}
  $ \text {Нормы скорости при } \mu=10^{-3}$
\end{center}
\vfill
\subsubsection{Вывод}
Системы являются консервативными при любых параметрах
\subsubsection{Графики}
\begin{figure}[H]
\center{\includegraphics[scale=1]{a201.txt.png}}
\caption{График при $\mu=10^{-1}$}
\label{fig:image}
\end{figure}

\begin{figure}[H]
\center{\includegraphics[scale=1]{a2001.txt.png}}
\caption{График при $\mu=10^{-2}$}
\label{fig:image}
\end{figure}

\begin{figure}[H]
\center{\includegraphics[scale=1]{a20001.txt.png}}
\caption{График при $\mu=10^{-3}$}
\label{fig:image}
\end{figure}

\subsection{Динамика процессов}
Рассмотрим $\mu = 0.1, \tau = 0.001, h = 0.05.$
\subsubsection{Первая задача}

\begin{figure}[H]
	\centering
	\begin{minipage}{.5\textwidth}
		\centering
		\includegraphics[scale=0.5]{0.png}
	\end{minipage}%
\end{figure}
\begin{figure}[H]
	\centering
	\begin{minipage}{.5\textwidth}
		\centering
		\includegraphics[scale=0.5]{1.png}
	\end{minipage}%
	\begin{minipage}{.5\textwidth}
		\centering
		\includegraphics[scale=0.5]{2.png}
	\end{minipage}
\end{figure}

\begin{figure}[H]
	\centering
	\begin{minipage}{.5\textwidth}
		\centering
		\includegraphics[scale=0.5]{3.png}
	\end{minipage}%
	\begin{minipage}{.5\textwidth}
		\centering
		\includegraphics[scale=0.5]{4.png}
	\end{minipage}
\end{figure}

\begin{figure}[H]
	\centering
	\begin{minipage}{.5\textwidth}
		\centering
		\includegraphics[scale=0.5]{5.png}
	\end{minipage}%
	\begin{minipage}{.5\textwidth}
		\centering
		\includegraphics[scale=0.5]{6.png}
	\end{minipage}
\end{figure}

\begin{figure}[H]
	\centering
	\begin{minipage}{.5\textwidth}
		\centering
		\includegraphics[scale=0.5]{7.png}
	\end{minipage}%
	\begin{minipage}{.5\textwidth}
		\centering
		\includegraphics[scale=0.5]{8.png}
	\end{minipage}
\end{figure}

\begin{figure}[H]
	\centering
	\begin{minipage}{.5\textwidth}
		\centering
		\includegraphics[scale=0.5]{9.png}
	\end{minipage}%
	\begin{minipage}{.5\textwidth}
		\centering
		\includegraphics[scale=0.5]{10.png}
	\end{minipage}
\end{figure}

\begin{figure}[H]
	\centering
	\begin{minipage}{.5\textwidth}
		\centering
		\includegraphics[scale=0.5]{20.png}
	\end{minipage}%
	\begin{minipage}{.5\textwidth}
		\centering
		\includegraphics[scale=0.5]{40.png}
	\end{minipage}
\end{figure}

\begin{figure}[H]
	\centering
	\begin{minipage}{.5\textwidth}
		\centering
		\includegraphics[scale=0.5]{60.png}
	\end{minipage}%
	\begin{minipage}{.5\textwidth}
		\centering
		\includegraphics[scale=0.5]{80.png}
	\end{minipage}
\end{figure}

\begin{figure}[H]
	\centering
	\begin{minipage}{.5\textwidth}
		\centering
		\includegraphics[scale=0.5]{100.png}
	\end{minipage}%
	\begin{minipage}{.5\textwidth}
		\centering
		\includegraphics[scale=0.5]{120.png}
	\end{minipage}
\end{figure}

\begin{figure}[H]
	\centering
	\begin{minipage}{.5\textwidth}
		\centering
		\includegraphics[scale=0.5]{141.png}
	\end{minipage}%
	\begin{minipage}{.5\textwidth}
		\centering
		\includegraphics[scale=0.5]{161.png}
	\end{minipage}
\end{figure}

\begin{figure}[H]
	\centering
	\begin{minipage}{.5\textwidth}
		\centering
		\includegraphics[scale=0.5]{181.png}
	\end{minipage}%
	\begin{minipage}{.5\textwidth}
		\centering
		\includegraphics[scale=0.5]{201.png}
	\end{minipage}
\end{figure}


\subsubsection{Вторая задача}
\begin{figure}[H]
	\centering
	\begin{minipage}{.5\textwidth}
		\centering
		\includegraphics[scale=0.5]{0sec.png}
	\end{minipage}%
	\begin{minipage}{.5\textwidth}
		\centering
		\includegraphics[scale=0.5]{1sec.png}
	\end{minipage}
\end{figure}

\begin{figure}[H]
	\centering
	\begin{minipage}{.5\textwidth}
		\centering
		\includegraphics[scale=0.5]{2sec.png}
	\end{minipage}%
	\begin{minipage}{.5\textwidth}
		\centering
		\includegraphics[scale=0.5]{3sec.png}
	\end{minipage}
\end{figure}

\begin{figure}[H]
	\centering
	\begin{minipage}{.5\textwidth}
		\centering
		\includegraphics[scale=0.5]{4sec.png}
	\end{minipage}%
	\begin{minipage}{.5\textwidth}
		\centering
		\includegraphics[scale=0.5]{5sec.png}
	\end{minipage}
\end{figure}

\begin{figure}[H]
	\centering
	\begin{minipage}{.5\textwidth}
		\centering
		\includegraphics[scale=0.5]{6sec.png}
	\end{minipage}%
	\begin{minipage}{.5\textwidth}
		\centering
		\includegraphics[scale=0.5]{7sec.png}
	\end{minipage}
\end{figure}

\begin{figure}[H]
	\centering
	\begin{minipage}{.5\textwidth}
		\centering
		\includegraphics[scale=0.5]{8sec.png}
	\end{minipage}%
	\begin{minipage}{.5\textwidth}
		\centering
		\includegraphics[scale=0.5]{9sec.png}
	\end{minipage}
\end{figure}

\begin{figure}[H]
	\centering
	\begin{minipage}{.5\textwidth}
		\centering
		\includegraphics[scale=0.5]{10sec.png}
	\end{minipage}%
	\begin{minipage}{.5\textwidth}
		\centering
		\includegraphics[scale=0.5]{39sec.png}
	\end{minipage}
\end{figure}

\begin{figure}[H]
	\centering
	\begin{minipage}{.5\textwidth}
		\centering
		\includegraphics[scale=0.5]{78sec.png}
	\end{minipage}%
	\begin{minipage}{.5\textwidth}
		\centering
		\includegraphics[scale=0.5]{117sec.png}
	\end{minipage}
\end{figure}

\begin{figure}[H]
	\centering
	\begin{minipage}{.5\textwidth}
		\centering
		\includegraphics[scale=0.5]{156sec.png}
	\end{minipage}%
	\begin{minipage}{.5\textwidth}
		\centering
		\includegraphics[scale=0.5]{196sec.png}
	\end{minipage}
\end{figure}

\begin{figure}[H]
	\centering
	\begin{minipage}{.5\textwidth}
		\centering
		\includegraphics[scale=0.5]{235sec.png}
	\end{minipage}%
	\begin{minipage}{.5\textwidth}
		\centering
		\includegraphics[scale=0.5]{274sec.png}
	\end{minipage}
\end{figure}

\begin{figure}[H]
	\centering
	\begin{minipage}{.5\textwidth}
		\centering
		\includegraphics[scale=0.5]{313sec.png}
	\end{minipage}%
	\begin{minipage}{.5\textwidth}
		\centering
		\includegraphics[scale=0.5]{353sec.png}
	\end{minipage}
\end{figure}

\begin{figure}[H]
	\centering
	\begin{minipage}{.5\textwidth}
		\centering
		\includegraphics[scale=0.5]{392sec.png}
	\end{minipage}%
\end{figure}

\subsection{Цикличность решения}
Рассмотрим $\mu = 0.1, \tau = 0.001, h = 0.05.$


\subsubsection{Первая задача}
Графики для V:
\begin{figure}[H]

	\centering
	\begin{minipage}{.5\textwidth}
		\centering
		\includegraphics[scale=0.5]{V_1_0.010.00010.1.png}
	\end{minipage}%
	\begin{minipage}{.5\textwidth}
		\centering
		\includegraphics[scale=0.5]{V_1_0.010.00010.01.png}
	\end{minipage}
\end{figure}


\begin{figure}[H]
	\centering
	\begin{minipage}{.5\textwidth}
		\centering
		\includegraphics[scale=0.5]{V_1_0.010.00010.001.png}
	\end{minipage}%
\end{figure}

Графики для H:
\begin{figure}[H]

	\centering
	\begin{minipage}{.5\textwidth}
		\centering
		\includegraphics[scale=0.5]{H_1_0.010.00010.1.png}
	\end{minipage}%
	\begin{minipage}{.5\textwidth}
		\centering
		\includegraphics[scale=0.5]{H_1_0.010.00010.01.png}
	\end{minipage}
\end{figure}


\begin{figure}[H]
	\centering
	\begin{minipage}{.5\textwidth}
		\centering
		\includegraphics[scale=0.5]{H_1_0.010.00010.001.png}
	\end{minipage}%
\end{figure}

\subsubsection{Вторая задача}
Графики для V:
\begin{figure}[H]

	\centering
	\begin{minipage}{.5\textwidth}
		\centering
		\includegraphics[scale=0.5]{V_2_0.010.00010.1.png}
	\end{minipage}%
	\begin{minipage}{.5\textwidth}
		\centering
		\includegraphics[scale=0.5]{V_2_0.010.00010.01.png}
	\end{minipage}
\end{figure}


\begin{figure}[H]
	\centering
	\begin{minipage}{.5\textwidth}
		\centering
		\includegraphics[scale=0.5]{V_2_0.010.00010.001.png}
	\end{minipage}%
\end{figure}

Графики для H:
\begin{figure}[H]

	\centering
	\begin{minipage}{.5\textwidth}
		\centering
		\includegraphics[scale=0.5]{H_2_0.010.00010.1.png}
	\end{minipage}%
	\begin{minipage}{.5\textwidth}
		\centering
		\includegraphics[scale=0.5]{H_2_0.010.00010.01.png}
	\end{minipage}
\end{figure}


\begin{figure}[H]
	\centering
	\begin{minipage}{.5\textwidth}
		\centering
		\includegraphics[scale=0.5]{H_2_0.010.00010.001.png}
	\end{minipage}%
\end{figure}

\subsubsection{Вывод}
При уменьшении параметра $\mu$ рисунок становится более "четким",
что свидетельствует о большем времени стабилизации. Период колебаний не зависит от параметра.
\section{Задача о стабилизации осциллирующей функции}
 \subsection{Постановка задачи 1}
Для системы (*) зададим начальные и граничные условия, которые определяются следующим образом:
$$
\left\{\begin{array}{l}
\rho_{0}(x)=2+\sin K \pi x, \quad x \in[0,1] \\
u_{0}(x)=0, \quad x \in[0,1] \\
u(t, 0)=u(t, 1)=0, \quad t \in[0, T]
\end{array}\right.
$$
Область $\Omega=[0, T] \times[0,1],$ функции $f$ и $f_{0}$ тождественно равны $0,$ параметр $K \in \mathrm{N}$ и удовлетворяет неравенству $1 \leq K \leq \frac{M}{10}$
Вычисления будут проводится до времени $N_{0} \tau,$ при котором решение перестанет зависеть от времени (выйдет на стационар). Критерием выхода на стационар будем считать $\left|V_{m}^{N_{0}}\right|<\varepsilon \quad m=0, \ldots, M$
\subsection{Постановка задачи 2}
Для системы (*) зададим начальные и граничные условия, которые определяются следующим образом:
$$
\left\{\begin{array}{l}
\rho_{0}(x)=1, \quad x \in[0,1] \\
u_{0}(x)=\sin K \pi x, \quad x \in[0,1] \\
u(t, 0)=u(t, 1)=0, \quad t \in[0, T]
\end{array}\right.
$$
Область $\Omega=[0, T] \times[0,1]$, функции $f$ и $f_{0}$ тождественно равны $0,$ параметр $K \in \mathrm{N}$ и удовлетворяет неравентсву $1 \leq K \leq \frac{M}{10}$
Вычисления будут проводится до времени $N_{0} \tau,$ при котором репение перестанет зависеть от времени (выйдет на стационар). Критерием выхода на стационар будем считать $\left|V_{m}^{N_{0}}\right|<\varepsilon \quad m=0, \ldots, M$
\subsection{Численные эксперименты}
Зафиксируем $h=0.01, \tau=0.0001$ и $\varepsilon=0.00001 .$ В таблицах ниже представлены значения $N_{0} \tau$ для различных
параметров системы.
\subsubsection{Задача 1}
\begin{center}
  \begin{tabular}{| l | l | l | l |}
    \hline
      \backslashbox{$K$}{$\mu$} & 0.1 & 0.01 & 0.001 \\ \hline
1 & 5.9418 & 57.8069 & 415.725 \\ \hline
2 & 21.3714 & 171.44 & 1242.1 \\ \hline
3 & 5.4567 & 50.5228 & 364.269 \\ \hline
4 & 18.4249 & 158.953 & 1242.22 \\ \hline
5 & 4.7458 & 48.2973 & 353.583 \\ \hline
6 & 16.9518 & 157.482 & 1226.79 \\ \hline
7 & 4.0285 & 45.529 & 349.287 \\ \hline
8 & 16.2146 & 156.744 & 1206.91 \\ \hline
9 & 3.6687 & 43.4275 & 346.286 \\ \hline
10 & 14.7411 & 155.271 & 1195.86 \\ \hline
  \end{tabular}
  \\
  $ \text {Время выхода на стационар}$
\end{center}
\vfill

\subsubsection{Задача 2}
\begin{center}
  \begin{tabular}{| l | l | l | l |}
    \hline
      \backslashbox{$K$}{$\mu$} & 0.1 & 0.01 & 0.001 \\ \hline
1 & 11.4168 & 92.4641 & 693.503 \\ \hline
2 & 2.7947 & 24.6897 & 188.722 \\ \hline
3 & 1.3342 & 11.3949 & 189.043 \\ \hline
4 & 0.8235 & 6.6491 & 50.2164 \\ \hline
5 & 0.7309 & 4.3085 & 32.5727 \\ \hline
6 & 0.6673 & 3.313 & 45.9372 \\ \hline
7 & 0.5511 & 3.3294 & 108.556 \\ \hline
8 & 0.5589 & 1.7483 & 12.8682 \\ \hline
9 & 0.5855 & 2.0309 & 72.2112 \\ \hline
10 & 0.5969 & 1.1484 & 8.2721 \\ \hline
  \end{tabular}
  \\
  $ \text {Время выхода на стационар }$
\end{center}
\vfill

\section{Задача "протекания"}
\subsection{Постановка задачи}
Для системы (*) зададим начальные и граничные условия, которые определяются следующим образом:
$$
\left\{\begin{array}{l}
\rho_{0}(x)=1, \quad x \in[0,10] \\
u_{0}(x)=0, \quad x \in[0,10] \\
u(t, 0)=v \quad t \in[0, T] \\
\rho(t, 0)=\tilde{\rho} \quad t \in[0, T] \\
\left.\frac{\partial u}{\partial x}\right|_{x=X}=0 \quad t \in[0, T]
\end{array}\right.
$$
Область $\Omega=[0, T] \times[0,10],$ а функции $f$ и $f_{0}$ тождественно равны $0 .$ Параметры $v(v>0)$ и $\tilde{\rho} \tilde{\rho} \geq 1$ задают скорость и плотность, "набегающего" потока.

Вычисления будут проводится до времени $N_{0} \tau,$ при котором решение перестанет зависеть от времени (выйдет на стационар). Критерием выхода на стационар будем считать $\left|V_{m}^{N_{0}}-V_{m}^{N_{0}-50}\right|<0.000001 \quad m=$ $0, \ldots, M$
\subsection{Численные эксперименты}
\begin{center}
  \begin{tabular}{| l | l | l | l | l |}
    \hline
      \backslashbox{$\tilde{\rho}$}{$v$} & 1 & 2 & 3 & 4  \\ \hline
1  & 145.175 & 48.8773 & 14.9237 & 8.6546 \\ \hline
2  & 139.249 & 43.764 & 13.7317 & 8.2277 \\ \hline
3  & 137.243 & 42.4554 & 13.4933 & 8.1338 \\ \hline
4  & 126.653 & 43.3988 & 13.5558 & 8.1328 \\ \hline
  \end{tabular}
  $ \text {Время выхода на стационар при  } \mu=10^{-1}$
\end{center}
\vfill

\begin{center}
  \begin{tabular}{| l | l | l | l | l |}
    \hline
      \backslashbox{$\tilde{\rho}$}{$v$}  & 1 & 2 & 3 & 4  \\ \hline
1  & 126.685 & 48.5548 & 14.8048 & 8.5819 \\ \hline
2  & 138.541 & 43.162 & 13.6477 & 8.1927 \\ \hline
3  & 136.108 & 41.7945 & 13.4121 & 8.0996 \\ \hline
4  & 123.693 & 42.8034 & 13.4841 & 8.1008 \\ \hline
  \end{tabular}
  $ \text {Время выхода на стационар при  } \mu=10^{-2}$
\end{center}
\vfill

\begin{center}
  \begin{tabular}{| l | l | l | l | l |}
    \hline
      \backslashbox{$\tilde{\rho}$}{$v$} & 1 & 2 & 3 & 4  \\ \hline
1  & 94.4752 & 47.9779 & 14.7463 & 8.529 \\ \hline
2  & 129.335 & 40.1225 & 13.2128 & 7.9827 \\ \hline
3  & 124.487 & 38.479 & 12.9478 & 7.8673 \\ \hline
4  & 89.0162 & 39.328 & 13.0656 & 7.8747 \\ \hline
  \end{tabular}
  $ \text {Время выхода на стационар при} \mu=10^{-3}$
\end{center}
\vfill

\subsection{Вывод}
Не замечено зависимости времени стабилизации от параметра $\mu$, при фиксированном  $\rho$ при увеличении $v$ время стабилизации уменьшается.
\subsection{Графики решения при стабилизации}
\begin{figure}[H]

	\centering
	\begin{minipage}{.5\textwidth}
		\centering
		\includegraphics[scale=0.5]{fourth110.001.png}
	\end{minipage}%
	\begin{minipage}{.5\textwidth}
		\centering
		\includegraphics[scale=0.5]{fourth110.01.png}
	\end{minipage}
\end{figure}

\begin{figure}[H]

	\centering
	\begin{minipage}{.5\textwidth}
		\centering
		\includegraphics[scale=0.5]{fourth110.1.png}
	\end{minipage}%
	\begin{minipage}{.5\textwidth}
		\centering
		\includegraphics[scale=0.5]{fourth120.001.png}
	\end{minipage}
\end{figure}

\begin{figure}[H]

	\centering
	\begin{minipage}{.5\textwidth}
		\centering
		\includegraphics[scale=0.5]{fourth120.01.png}
	\end{minipage}%
	\begin{minipage}{.5\textwidth}
		\centering
		\includegraphics[scale=0.5]{fourth120.1.png}
	\end{minipage}
\end{figure}

\begin{figure}[H]

	\centering
	\begin{minipage}{.5\textwidth}
		\centering
		\includegraphics[scale=0.5]{fourth130.001.png}
	\end{minipage}%
	\begin{minipage}{.5\textwidth}
		\centering
		\includegraphics[scale=0.5]{fourth130.01.png}
	\end{minipage}
\end{figure}

\begin{figure}[H]

	\centering
	\begin{minipage}{.5\textwidth}
		\centering
		\includegraphics[scale=0.5]{fourth130.1.png}
	\end{minipage}%
	\begin{minipage}{.5\textwidth}
		\centering
		\includegraphics[scale=0.5]{fourth140.001.png}
	\end{minipage}
\end{figure}

\begin{figure}[H]

	\centering
	\begin{minipage}{.5\textwidth}
		\centering
		\includegraphics[scale=0.5]{fourth140.01.png}
	\end{minipage}%
	\begin{minipage}{.5\textwidth}
		\centering
		\includegraphics[scale=0.5]{fourth140.1.png}
	\end{minipage}
\end{figure}


\begin{figure}[H]

	\centering
	\begin{minipage}{.5\textwidth}
		\centering
		\includegraphics[scale=0.5]{fourth210.001.png}
	\end{minipage}%
	\begin{minipage}{.5\textwidth}
		\centering
		\includegraphics[scale=0.5]{fourth210.01.png}
	\end{minipage}
\end{figure}

\begin{figure}[H]

	\centering
	\begin{minipage}{.5\textwidth}
		\centering
		\includegraphics[scale=0.5]{fourth210.1.png}
	\end{minipage}%
	\begin{minipage}{.5\textwidth}
		\centering
		\includegraphics[scale=0.5]{fourth220.001.png}
	\end{minipage}
\end{figure}

\begin{figure}[H]

	\centering
	\begin{minipage}{.5\textwidth}
		\centering
		\includegraphics[scale=0.5]{fourth220.01.png}
	\end{minipage}%
	\begin{minipage}{.5\textwidth}
		\centering
		\includegraphics[scale=0.5]{fourth220.1.png}
	\end{minipage}
\end{figure}



\begin{figure}[H]

	\centering
	\begin{minipage}{.5\textwidth}
		\centering
		\includegraphics[scale=0.5]{fourth230.001.png}
	\end{minipage}%
	\begin{minipage}{.5\textwidth}
		\centering
		\includegraphics[scale=0.5]{fourth230.01.png}
	\end{minipage}
\end{figure}

\begin{figure}[H]

	\centering
	\begin{minipage}{.5\textwidth}
		\centering
		\includegraphics[scale=0.5]{fourth230.1.png}
	\end{minipage}%
	\begin{minipage}{.5\textwidth}
		\centering
		\includegraphics[scale=0.5]{fourth240.001.png}
	\end{minipage}
\end{figure}

\begin{figure}[H]

	\centering
	\begin{minipage}{.5\textwidth}
		\centering
		\includegraphics[scale=0.5]{fourth240.01.png}
	\end{minipage}%
	\begin{minipage}{.5\textwidth}
		\centering
		\includegraphics[scale=0.5]{fourth240.1.png}
	\end{minipage}
\end{figure}



\begin{figure}[H]

	\centering
	\begin{minipage}{.5\textwidth}
		\centering
		\includegraphics[scale=0.5]{fourth310.001.png}
	\end{minipage}%
	\begin{minipage}{.5\textwidth}
		\centering
		\includegraphics[scale=0.5]{fourth310.01.png}
	\end{minipage}
\end{figure}

\begin{figure}[H]

	\centering
	\begin{minipage}{.5\textwidth}
		\centering
		\includegraphics[scale=0.5]{fourth310.1.png}
	\end{minipage}%
	\begin{minipage}{.5\textwidth}
		\centering
		\includegraphics[scale=0.5]{fourth320.001.png}
	\end{minipage}
\end{figure}

\begin{figure}[H]

	\centering
	\begin{minipage}{.5\textwidth}
		\centering
		\includegraphics[scale=0.5]{fourth320.01.png}
	\end{minipage}%
	\begin{minipage}{.5\textwidth}
		\centering
		\includegraphics[scale=0.5]{fourth320.1.png}
	\end{minipage}
\end{figure}





\begin{figure}[H]

	\centering
	\begin{minipage}{.5\textwidth}
		\centering
		\includegraphics[scale=0.5]{fourth330.001.png}
	\end{minipage}%
	\begin{minipage}{.5\textwidth}
		\centering
		\includegraphics[scale=0.5]{fourth330.01.png}
	\end{minipage}
\end{figure}

\begin{figure}[H]

	\centering
	\begin{minipage}{.5\textwidth}
		\centering
		\includegraphics[scale=0.5]{fourth330.1.png}
	\end{minipage}%
	\begin{minipage}{.5\textwidth}
		\centering
		\includegraphics[scale=0.5]{fourth340.001.png}
	\end{minipage}
\end{figure}

\begin{figure}[H]

	\centering
	\begin{minipage}{.5\textwidth}
		\centering
		\includegraphics[scale=0.5]{fourth340.01.png}
	\end{minipage}%
	\begin{minipage}{.5\textwidth}
		\centering
		\includegraphics[scale=0.5]{fourth340.1.png}
	\end{minipage}
\end{figure}






\begin{figure}[H]

	\centering
	\begin{minipage}{.5\textwidth}
		\centering
		\includegraphics[scale=0.5]{fourth410.001.png}
	\end{minipage}%
	\begin{minipage}{.5\textwidth}
		\centering
		\includegraphics[scale=0.5]{fourth410.01.png}
	\end{minipage}
\end{figure}

\begin{figure}[H]

	\centering
	\begin{minipage}{.5\textwidth}
		\centering
		\includegraphics[scale=0.5]{fourth410.1.png}
	\end{minipage}%
	\begin{minipage}{.5\textwidth}
		\centering
		\includegraphics[scale=0.5]{fourth420.001.png}
	\end{minipage}
\end{figure}

\begin{figure}[H]

	\centering
	\begin{minipage}{.5\textwidth}
		\centering
		\includegraphics[scale=0.5]{fourth420.01.png}
	\end{minipage}%
	\begin{minipage}{.5\textwidth}
		\centering
		\includegraphics[scale=0.5]{fourth420.1.png}
	\end{minipage}
\end{figure}





\begin{figure}[H]

	\centering
	\begin{minipage}{.5\textwidth}
		\centering
		\includegraphics[scale=0.5]{fourth430.001.png}
	\end{minipage}%
	\begin{minipage}{.5\textwidth}
		\centering
		\includegraphics[scale=0.5]{fourth430.01.png}
	\end{minipage}
\end{figure}

\begin{figure}[H]

	\centering
	\begin{minipage}{.5\textwidth}
		\centering
		\includegraphics[scale=0.5]{fourth430.1.png}
	\end{minipage}%
	\begin{minipage}{.5\textwidth}
		\centering
		\includegraphics[scale=0.5]{fourth440.001.png}
	\end{minipage}
\end{figure}

\begin{figure}[H]

	\centering
	\begin{minipage}{.5\textwidth}
		\centering
		\includegraphics[scale=0.5]{fourth440.01.png}
	\end{minipage}%
	\begin{minipage}{.5\textwidth}
		\centering
		\includegraphics[scale=0.5]{fourth440.1.png}
	\end{minipage}
\end{figure}
\end{document}

