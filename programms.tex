\documentclass[12pt]{article}
\usepackage[T2A]{fontenc}
\usepackage[utf8]{inputenc}
\usepackage{amsmath}
\usepackage[english, russian]{babel}
\usepackage{listings}


\begin{document}
\begin{center} \large \bf
Мехмат. Сошников Владимир, группа 410.\\
Численное моделирование нестационарного одномерного течения газа с использованием неявной последовательной разностной схемы А.Г. Соколова   ($u$, $\rho$). 


\section{Постановка задачи}

В данной работе будет рассматриваться разностная схема А.Г. Соколова для решения начально - краевых задач для системы уравнений, описывающей нестационарное одномерное движение вязкого баротропного газа:

\begin{equation*}
 \begin{cases}
   \text
     {
       $
         \frac {\partial \rho} {\partial t} + 
         \frac {\partial \rho u} {\partial x} = 
         0
       $
     }
   \\
   \text 
     {
       $
          \rho \frac {\partial u} {\partial t} +
          \rho u \frac {\partial u} {\partial x} + 
          \frac {\partial p} {\partial x} =
          \mu \frac {\partial^{2} u} {\partial x ^{2}} +
          \rho f
       $
     }
   \\
   \text 
     {
       $
         p = p (\rho)
       $
     } 
   \\
   \text 
     {
       $
        (\rho, u)|_{t = 0} = (\rho_{0}, \,u_{0}), \,  x \in [0, X]
       $
     } 
   \\
   \text 
     {
       $
        u (t, 0) = u (t, X) = 0, \,t \in [0, T]
       $
     } 
 \end{cases}
\end{equation*}
Неизвестные функции: плотность $\rho$ и скорость $u$ являются функциями переменных Эйлера
$$
(t,x) \in Q = [0,T] \times [0,X].
$$
В уравнения входят еще две неизвестные функции - давление$p$, зависящее от плотности и вектор внешних сил $f$, являющийся функцией переменных Эйлера.
В данной работе рассматривается зависимость $p = \rho ^{\gamma}$, где $\gamma = 1.4$
\section{Описание алгоритма}
\subsection{Используемые обозначения}
Рассмотрим временной интервал $[0,T]$, введем на нем равномерную сетку:
$ \bar w_{\tau} = \{n \tau | n = 0,...,N\}$, $N \tau = T$
Рассмотрим пространственную область в виде отрезка $\bar \Omega = [0, X]$. Обозначим  через $\bar w_{h} = \{mh|m = 0,...,M\}$ (где $Mh = X$), равномерную сетку с шагом $h$, а
через $w_{h}^{1/2} = \{mh + \frac{h}{2} | m = 0,..., M - 1\}$ сдвинутую равномерную сетку с полуцелыми узлами.
Введем также обозначения для сеток:
$
\bar Q_{\tau h} := \bar \omega_{\tau} \times \bar \omega_{h}
$
и 
$
Q_{\tau h}^{1/2} = \bar \omega_{\tau} \times \omega_{h}^{1/2}
$


Значение функции $g$ в узле (m, n) будем обозначать через $g_{n}^{m}$
Для сокращения записей будем также использовать следующие обозначения:
\\
$
g_{m}^{n+1} = \hat g \
$,
$
g_{m \pm 1} ^{n} = g^{\pm 1}
$

Обозначения для среднего значения величин сеточной функции в двух соседних узлах: \\
$
g_{s} = \frac{g_{m+1}^{n} + g_{m}^{n}}{2}
$,
$
g_{\bar s} = \frac {g_{m}^{n} + g_{m-1}^{n}}{2}
$
\\
Для разностных операторов применяются следующие обозначения:
\\
$$
g_{t}=\frac{g_{m}^{n+1}-g_{m}^{n}}{\tau}, \quad g_{x}=\frac{g_{m+1}^{n}-g_{m}^{n}}{h},  \quad g_{\bar{x}}=\frac{g_{m}^{n}-g_{m-1}^{n}}{h}, \quad g_{x \bar{x}}=\frac{g_{m-1}^{n}-2 g_{m}^{n}+g_{m+1}^{n}}{h^{2}}
$$
Для приближения конвективных слагаемых в дифференциальных операторах будут использоваться разностные аппроксимации 
против потока. Для этих выражений введем обозначение:
$$
\delta\{W, V\}=\frac{V+|V|}{2} W_{\bar{x}}+\frac{V-|V|}{2} W_{x}=\left\{\begin{array}{l}
V W_{\bar{x}}, \text { если } V \geq 0 \\
V W_{x}, \text { если } V<0
\end{array}\right.
$$
В данной схеме в конвективных слагаемых узел шаблона, в котором нужно брать значение сеточной функции $F$,
 зависит от знака компоненты вектора скорости $V$. Для этих выражений используется обозначение:
$$
\sigma\{F, V\}=F \frac{|V|-V}{2|V|}+F^{(-1)} \frac{|V|+V}{2|V|}=\left\{\begin{array}{l}
F, \text { ecли } V<0 \\
F^{(-1)}, \text {если } V \geq 0
\end{array}\right.
$$

\subsection{Схема в веденных обозначениях}
ыы
\subsection{Поточечная запись схемы}
\end{center}
\end{document}
